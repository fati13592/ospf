\documentclass[a4paper,11pt]{article}
\usepackage[T1]{fontenc}
\usepackage[utf8]{inputenc}
\usepackage{lmodern}
\usepackage[pdftex]{graphicx}
\DeclareGraphicsExtensions{.jpg}
%\usepackage[spanish]{babel}

%http://www.slideshare.net/navarrojavier22/redes-y-conectividad-enrutamiento-y-protocolos-de-enrutamiento-ppts?qid=c4833f9d-6aa3-47de-ae99-ad85b512551e&v=default&b=&from_search=1

%http://www.slideshare.net/FranciscoNvoaManuel/ccna-routing-switching-novedades-enrutamiento-ospf-multirea?qid=c4833f9d-6aa3-47de-ae99-ad85b512551e&v=default&b=&from_search=4#

\title{Protocolos de enrutamiento OSPF}
\author{Fátima Armenteros López}
\date{Mayo 2014}

\begin{document}

\maketitle
\thispagestyle{empty}
\titlepage
\tableofcontents

\newpage
\section{Introducción}\label{sec:intro}
\paragraph{  }
  Los protocolos de enrutamiento IP funcionan en un conjunto de routers, enviando mensajes a sus routers vecinos para ayudar a estos a aprender las mejores rutas a cada subred. 
  
  A continuación explicaré el funcionamiento de los distintos protocolos centrandome en OSPF.


\section{Enrutamiento}\label{sec:en}
\paragraph{  }
Es el proceso de seleccionar los mejores caminos en una red, es algo fundamental, ya que transfiere información de origen a destino. Se usa enrutamiento no sólo para internet si no también en distintos tipos de redes, como la telefónica.

El enrutamiento IP depende de las reglas del direccionamiento IP, siendo uno de sus objetivos principales crear un enrutamiento eficiente.

Los routers utilizan tres métodos para añadir rutas a sus tablas de enrutamiento: rutas conectadas directamente, rutas estáticas y protocolos de enrutamiento dinámico. Los routers siempre añaden las rutas conectadas cuando las interfaces están configuradas y funcionando.  

\subsection{Enrutamiento estático}\label{subsec:enestatico}
\paragraph{  }
El enrutamiento estático consiste en añadir rutas a un router manualmente, sin usar protocolos de enrutamiento dinámico. Esto implica que el mantenimiento de la tabla de enrutamiento debe hacerse manualmente.

Es el método menos usado.

\begin{description}
  \item[Ventajas]
  Las rutas estáticas no requieren la misma cantidad de procesamiento y sobrecarga que requieren los protocolos de enrutamiento dinámico.
  Son más fáciles de configuar y entender.
  \item[Inconvenientes]
  Cada vez que haya un cambio en la red el administrador deberá cambiar la tabla de enrutamiento manualmente.
  No es viable en el caso de que la red sea muy grande.
  Es más probable que se produzcan errores en la configuración.
  Se requiere un gran conocimiento de la red. 
\end{description}

\subsection{Enrutamiento dinámico}\label{subsec:endinamico}
\paragraph{  }
  Las rutas son aprendidas mediante protocolos de enrutamiento dinámico, los cuales definen varias formas en que los routers dialogan entre ellos para determinar las mejores rutas a cada destino.
  
  Un protocolo de enrutamiento es un conjunto de mensajes, reglas y algoritmos utilizados por los routers para el aprendizaje de rutas.  
  
  \begin{description}
  \item[Ventajas]
  Se necesitan menos intervenciones del administrador de la red.
  La configuración es menos propensa a errores.
  Es mucho mas sencillo que la red crezca.
  Los protocolos reaccionan automáticamente a los cambios.
  \item[Inconvenientes]
  Utilizan más recursos de los routers.
  Se necesitan más conocimientos para la configuración y resolución de problemas.
\end{description}

\subsubsection{Enrutamiento interno (IGP)}
\paragraph{  }
  Fue diseñado y pensado para el uso en un único sistema autónomo(AS)\footnote{Un AS es una internetwork bajo el control administrativo de una única organización}.
  
  Los protocolos eligen la mejor ruta a una subred basándose en las métricas. 
  
\begin{itemize}
  \item \textbf{Vector distancia}\\
  RIP fue el primer protocolo por vector distancia utilizado y después se introdujo IGRP, propietario de Cisco. Las actualizaciones son periódicas.  
  
  Más tarde Cisco creó el protocolo EIGRP que tiene características de IGRP y soluciona algunos problemas, fue llamado un protocolo híbrido. 
  
  \item \textbf{Estado del enlace}\\
  Se utiliza en algunos de los protocolos definidos más recientemente (OSPF, IS-IS). Las actualizaciones se producen por eventos.
  
  Solucionaron los principales problemas de los protocolos por vector distancia.
  
\end{itemize}

  \begin{description}
    \item[Convergencia] Los protocolos por estado de enlace son más rápidos.
    \item[CPU y RAM] Los protocolos por estado de enlace consumen mucha más memoria.
    \item[Prevención de bucles] Los protocolos por vector distancia necesitan de características adicionales para evitarlos.
    \item[Diseño] Los protocolos por vector distancia no necesitan mucha planificación.
    \item[Configuración] Los protocolos por vector distancia, normalmente, necesitan menos configuración.
  \end{description}


\subsubsection{Enrutamiento externo (EGP)}
\paragraph{  }
Fue diseñado y pensado para el uso entre diferentes sistemas autónomos.

La toma de decisiones está basada en políticas.

Actualmente sólo existe un EGP legítimo: BGP.


\section{Open Shortest Path First(OSPF)}\label{sec:ospf}
\paragraph{  }
Es un protocolo de enrutamiendo de estado enlace no propietario y que proporciona gran escalabilidad.

Sus características se podrían dividir en: vecinos, intercambio de bases de datos y cálculo de rutas. En primer lugar se forma una relación de vecindad entre los routers que sirve como base para todas las comunicaciones. Cuando ya se han hecho vecinos, intercambian el contenido de sus LSDB\footnote{Base de datos enlace-estado (link-state database)}. Una vez el router dispone de información en su LSDB, utiliza el algoritmo SPF de Dijkstra para calcular las mejores rutas y las añade a la tabla de enrutamiento.

Un router con OSPF debe contener una tabla de vecinos, una LSDB y una tabla de enrutamiento.

\subsection{Vecinos}
\paragraph{  }
Para un router, un vecino es otro router que se conecta al mismo enlace de datos y con el que debe intercambiar información de enrutamiento. Un router sabe cuando uno de sus vecinos está bien y en el momento que pierda la conexión con alguno debe recalcular las entradas de la tabla de enrutamiento.

Los vecinos necesitan saber que router les ha enviado cierto mensaje así que OSPF utiliza un ID de router(RID), que también es usado por la LSDB.

 Se pueden añadir nuevos vecinos a una red sin tener que reconfigarla gracias al proceso Hello.

\begin{itemize}
  \item \textbf{Búsqueda de vecinos diciendo Hello}\\
  Cuando el router ha seleccionado su RID y se activan las interfaces comienza a enviar paquetes Hello por multidifusión, esperando recibir paquetes Hello de otros routers.
  
  El mensaje Hello incluye el RID del remitente, el ID de zona, el intervalo de Hello, el intervalo muerto, la prioridad del router, el RID del router designado, el RID del router designado de respaldo y una lista de vecinos que el router remitente ya conoce.
  
  El router que ha recibido el Hello añade el RID del remitente a su tabla de vecinos, que enviará en el próximo Hello, en cuanto ve su propio RID en un Hello recibido asume que se ha establecido una comunicación bidireccional con ese vecino. Cuando se llega a ese punto ya se puede intercambiar información más detallada, como las LSAs\footnote{Publicación de estado del enlace (Link State Advertisement)}. 
  \item \textbf{Problemas}\\
  Para que dos routers se hagan vecinos es necesario que los siguientes campos coincidan:
  
  -La máscara de subred empleada en la subred
  
  -El número de subred
  
  -El intervalo de Hello
  
  -El intervalo muerto
  
  -El ID de área OSPF
  
  -Pasar las comprobaciones de autenticación en el caso de que se usen
  
  -El valor del indicador de área interna
  
  Los routers envían mensajes a intervalos marcados por el intervalo Hello y el intervalo muerto es el tiempo que esperan por un Hello antes de marcar a ese vecino como "down".
  
  \item \textbf{Estados de vecindad}\\
  Son la percepción que tiene un router de la cantidad de trabajo que se ha realizado en los procesos llevados a cabo con un vecino.
  
  Con los estados se puede saber fácilmente si un vecino funciona correctamente.
  
  -Down, cuando ya se conocía la existencia del vecino pero ha fallado la interfaz.
  
  -Init, cuando se activa la interfaz, la relación de vecindad se está inicializando.
  
  -Seen, cuando recibe un Hello con su RID y todos los parámetros son correctos.
  
  -Full, cuando los dos vecinos conocen los mismos datos de la LSBD y son completamente adyacentes.
  
  \item \textbf{Router designado(DR)}\\
  La elección de un DR sirve para que sólo se intercambie información topológica entre el DR y todos los demás routers, y no entre cada pareja de routers. Todas las actualizaciones de enrutamientos van desde y hacia el DR, el DR la distribuye a el resto de routers.
  
  Esto evita sobrecargar las subredes cuando hay muchos routers o , si estánrouters conectados a una LAN, evita toda la información redundante que se produciría.
  
  Si se pierde el DR seleccionado pueden ocurrir retardos en la convergencia, por lo que existe un router de respaldo (BDR) que puede hacer las veces de DR.
  
  Para elegir el DR los routers examinan los campos del paquete Hello y el que tenga el valor de prioridad más alto es el elegido, en caso de empate será el que tenga el RID más alto. Normalmente el que tenga la segunda priridad más alta es el designado para BDR.
  
\end{itemize}

\subsection{Intercambio de datos}
\paragraph{  }
Los routers OSPF intercambian el contenido de sus LSDB para que ambos vecinos tengan una copia exacta de la misma. Una vez que dos routers deciden intercambiar sus bases de datos, primero se envian sus listas de LSAs y de aquellas que el router no tenga una copia se la pide al vecino que se la enviará completa.

Cuando dos vecinos acaban este proceso se emplea el estado de vecindad Full. Cuando están en este estado continuan haciendo tareas de mantenimiento, siguen enviando mensajes Hello y en caso de que haya un cambio enviarán nuevas copias de las LSAs modificadas.


\subsection{Cálculo de la tabla de enrutamiento}
\paragraph{  }
Para rellenar la tabla de enrutamiento cada router ejecuta el algoritmo SPF de Dijsktra aplicado a la LSDB y selecciona las mejores rutas.

La LSDB está formada por listas de números de subredes y listas de routers con los enlaces a los que está conectado cada uno. Basándose en estos datos el algoritmo SPF cálcula la mejor ruta que va desde el router a las subredes de la LSDB.

Para cada posible ruta se suman los costes de las interfaces salientes y OSPF selecciona la de coste más bajo. 

En el caso de que se produzca un empate entre entre el coste de varias rutas hacia la misma subred, siendo este el mínimo, el router puede poner hasta 16 de ellas en la tabla de enrutamiento. Cuando esto ocurre pueden usarse a la vez todas las rutas para efectuar un equilibrado de la carga.

\begin{itemize}
  \item \textbf{Áreas de OSPF}\\
  Un sistema autónomo OSPF puede configurarse jerárquicamente en áreas, su uso resuelve algunos de los problemas que se dan cuando las redes son muy grandes.
  
  Cada área ejecuta su propio algoritmo de enrutamiento, con cada router de un área difundiendo su estado a todos los demás dentro de ese área. El router fronterizo(ABR) es el responsable de sacar los paquetes fuera del área.
  
  Habrá también una única área que actuará como área troncal (bakbone) cuya funcion principal es enrutar el tráfico entre las demás áreas del sistema autónomo.
\end{itemize}


\section{Configuración básica de OSPF}
\paragraph{  }
Condiguración básica de prueba en Packet Tracer sobre el siguiente escenario.
\begin{figure}[ht!]
  \begin{center}
    \includegraphics[width=\textwidth]{1.jpg}
    \label{fig:base}
    \caption{escenario base}
  \end{center}
\end{figure}


Cuando estén todas las interfaces de los routers configuradas y levantadas, lo primero de todo será acceder al modo de configuración de ospf y después simplemente habrá que activar ospf en cada una de las interfaces.

\begin{figure}[ht!]
  \begin{center}
    \includegraphics[width=\textwidth]{2.jpg}
    \label{fig:}
    \caption{ospf en el Router2}
  \end{center}
\end{figure}


\subsection{Comandos OSPF}
\paragraph{  }
Estos son todos los comandos que se podrían usar a la hora de configurar OSPF.

router(config)\# \textbf{router ospf} \textit{proceso} --> Accede al modo de configuración de OSPF

router(config-router)\# \textbf{network} \textit{direccion wildcard} \textbf{area} \textit{area-id} --> Habilita OSPF en las interfaces cuya configuración IP coincide con la dirección de red y máscara wildcard proporcionadas. Se establece un área de OSPF


router(config-router)\# \textbf{router id} \textit{valor\_id} --> Establece el valor de ID de OSPF

router(config-router)\# \textbf{default-information originate} --> Propaga la ruta estática por defecto en las actualizaciones OSPF

router(config-router)\# \textbf{auto-cost reference-bandwidth} \textit{valor} --> Establece el valor de referencia para los cálculos de coste de OSPF. Necesario en redes modernas de muy alta velocidad

router(config-if)\# \textbf{bandwidth} \textit{ancho-de-banda} --> Modifica el ancho de banda teórico disponible en una interfaz

router(config-if)\# \textbf{ip ospf cost} \textit{coste} --> Modifica el coste OSPF en un puerto específico

router(config-if)\# \textbf{ip ospf priority} \textit{valor} --> Establece la prioridad OSPF. Es un valor comprendido entre 0 y 255

router(config-if)\# \textbf{ip ospf hello-interval} \textit{segundos} --> Establece el intervalo de tiempo de los mensajes de saludo OSPF

router(config-if)\# \textbf{ip ospf dead-interval} \textit{segundos} --> Establece el intervalo de espera en OSPF

router\# \textbf{show ip ospf} --> Muestra información general sobre las instancias OSPF

router\# \textbf{show ip ospf interface} \textit{interfaz X/Y} --> Muestra información OSPF en la interfaz

router\# \textbf{show ip ospf neighbor} --> Muestra información y estado de los vecinos OSPF

router\# \textbf{show ip protocols} --> Muestra información sobre los protocolos de enrutamiento activos


%
%\begin{thebibliography}{99}
 % \bibitem{ccna}CCNA ICND2, Wendell Odom, CCIE\textregistered  No. 1624
%\end{thebibliography}



\end{document}
